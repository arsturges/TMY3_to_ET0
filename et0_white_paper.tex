%% LyX 1.6.7 created this file.  For more info, see http://www.lyx.org/.
%% Do not edit unless you really know what you are doing.
\documentclass[english]{article}
\usepackage[T1]{fontenc}
\usepackage[latin9]{inputenc}
\usepackage{amsmath}
\usepackage{amssymb}

\makeatletter

%%%%%%%%%%%%%%%%%%%%%%%%%%%%%% LyX specific LaTeX commands.
%% Because html converters don't know tabularnewline
\providecommand{\tabularnewline}{\\}

%%%%%%%%%%%%%%%%%%%%%%%%%%%%%% User specified LaTeX commands.
\makeatother

\makeatother

\usepackage{babel}

\begin{document}

\title{Computing Evapotranspiration Using TMY3 Weather Data}


\author{Andrew Sturges}

\maketitle

\section{Reference Evapotranspiration ($ET_{0}$)}

Evapotranspiration, expressed in units of water depth per unit time
(e.g. $\frac{mm}{day}$), is a combination of two separate processes
whereby water is lost on the one hand from the soil surface by evaporation
and on the other hand from the crop by transpiration \cite{FAO-56}.
For a given geographic location, a reference evapotranspiration, $ET_{0}$,
provides a relative indication of evapotranspiration losses, and thus,
irrigation needs. When multiplied by a crop coefficient $K_{c}$,
$ET_{0}$ yields the crop evapotranspiration $ET_{c}$, the actual
theoretical irrigation requirement of a given crop at a given location.

\begin{equation}
ET_{0}\ast K_{c}=ET_{c}\end{equation}



\subsection{The Penman-Monteith Equation}

$ET_{0}$ data for this report was calculated with the FAO Penman-Monteith
equation according to methods developed by the UN's Food and Agriculture
Organization (FAO) in Crop Evapotranspiration \cite{FAO-56}. \begin{equation}
\frac{0.408\Delta(R_{n}-G)+\gamma\frac{900}{T+273}u_{2}(e_{s}-e_{a})}{\Delta+\gamma(1+0.34u_{2})}=ET_{0}\label{eq:Penman-Monteith}\end{equation}


This equation depends on solar radiation, air temperature, air humidity
and wind speed data. See table \ref{Flo:Penman Equation Variables}
for an explanation of the variables. All input data for the FAO Penman-Monteith
Equation came from the National Renewable Energy Laboratory's (NREL)
National Solar Radiation Database, commonly known as TMY3.

%
\begin{table}
\caption{Penman-Monteith Equation Variables}


\label{Flo:Penman Equation Variables}

\begin{tabular}{|c|c|c|}
\hline 
Variable  & Description  & Units\tabularnewline
\hline
\hline 
$ET_{0}$  & Reference evapotranspiration  & $\frac{mm}{day}$\tabularnewline
\hline 
$R_{n}$  & Net radiation at the crop surface  & $\frac{MJ}{m^{2}day}$\tabularnewline
\hline 
$G$  & Soil heat flux density  & $\frac{MJ}{m^{2}day}$\tabularnewline
\hline 
$T$  & Mean daily air temperature at 2 meters height  & $\textdegree C$\tabularnewline
\hline 
$u_{2}$  & Wind speed at 2 meters height  & $\frac{m}{s}$\tabularnewline
\hline 
$e_{s}$  & Saturation vapor pressure  & $kPa$ \tabularnewline
\hline 
$e_{a}$  & Actual vapor pressure  & $kPa$\tabularnewline
\hline 
$e_{s}-e_{a}$  & Saturation vapor pressure deficit  & $kPa$\tabularnewline
\hline 
$\Delta$  & Slope of the vapor pressure curve  & $\frac{kPa}{\textdegree C}$\tabularnewline
\hline 
$\gamma$  & Psychrometric constant  & $\frac{kPa}{\textdegree C}$\tabularnewline
\hline
\end{tabular}
\end{table}


Most of these variables are composites of more fundamental data. These
data come from TMY3. 


\subsection{TMY3 Data }

TMY3 data come from 1020 weather stations around the U.S. which record
climate parameters once every hour. To account for variability across
years (e.g., a particular year's unseasonably high occurrence of
hurricanes in the south), NREL developed a statistical method of choosing
a reference year to represent each month of data in the TMY3 dataset
for each individual weather station \cite{wilcox2008users}. For example,
January weather data from Gainesville, Florida, (station ID 722146)
comes from 1994 records, while February data comes from 1991 records.
In this way, the TMY3 dataset aims to represent a typical meteorological
year (TMY). To arrive at the list of variables in table \ref{Flo:Penman Equation Variables}
requires calculations on more fundamental input data. These, which
come from TMY3, appear in table \ref{Flo:TMY3 data inputs}.

%
\begin{table}
\caption{Inputs from TMY3 Data}


\label{Flo:TMY3 data inputs}

\begin{tabular}{|c|c|c|c|c|}
\hline 
Variable  & Description  & Units  & Dependent $ET_{0}$ Variable  & TMY3 Field\tabularnewline
\hline
\hline 
$T_{max}$  & Average daily high temperature  & $\textdegree C$  & $e_{s}$  & 32\tabularnewline
\hline 
$T_{min}$  & Average daily low temperature  & $\textdegree C$  & $e_{s}$  & 32\tabularnewline
\hline 
$T_{dew}$  & Average dew point temperature  & $\textdegree C$  & $e_{a}$  & 32\tabularnewline
\hline 
$z$  & Elevation  & $m$  & $\gamma$  & 7 (header 1)\tabularnewline
\hline 
$u_{2}$  & Average wind speed  & $\frac{m}{s}$  & $u_{2}$  & 47\tabularnewline
\hline 
Latitude  & Latitude  & $degrees$  & $R_{n}$  & 5 (header 1)\tabularnewline
\hline 
$n$  & Average sunshine duration  & $hours$  & $R_{n}$  & 8\tabularnewline
\hline
\end{tabular}
\end{table}


Using these seven TMY3 variables, it is possible to compute all nine
terms which appear in the Penman-Monteith equation.


\subsection{Scope }

To estimate national crop yields, this study limits its scope to state-level
data. For TMY3 data, this means that primary data from all weather
stations in a given state are compiled and averaged by month to come
up with the terms for each $ET_{0}$ calculation. This averaging effectively
creates one hypothetical master weather station for each state, with
a representative latitude and elevation.

Because this study is primarily concerned with large-scale agriculture,
most of which is found at elevations below 1000 meters (reference?
Use a USDA database?), the calculation only includes weather stations
at or below an elevation of 1000 meters. Note that this excludes Wyoming
and Colorado from the data set becase all weather stations in those
two states are at elevations above 1000 meters. (I should confirm
this.)

Furthermore, because of the internal weather differences inside some
large states, we split these states by longitude or latitude as is
oppropriate. Table \ref{Flo:state_subregions} lists these subregions.

%
\begin{table}


\caption{State Subregions}


\label{Flo:state_subregions}

\begin{tabular}{|c|c|c|}
\hline 
State & Subregion & Border\tabularnewline
\hline
\hline 
Florida  & North/South & $27\textdegree$ North\tabularnewline
\hline 
Kansas & East/West & $98\textdegree$\tabularnewline
\hline 
North Dakota & East/West & $ $$100\textdegree$\tabularnewline
\hline 
South Dakota & East/West & $100\textdegree$\tabularnewline
\hline 
Nebraska & East/West & $100\textdegree$\tabularnewline
\hline 
Oklahoma & East/West & $98\textdegree$\tabularnewline
\hline 
Oregon & East/West & $120\textdegree$\tabularnewline
\hline 
Texas & East/West & $98\textdegree$\tabularnewline
\hline 
Washington & East/West & $120\textdegree$\tabularnewline
\hline
\end{tabular}
\end{table}



\subsection{TMY3 Variables }

Treatment of the FAO Penman-Monteith equation and derivation of its
constituent terms was done according to the methods laid out in Crop
evapotranspiration \cite{FAO-56}. However, the following variables
deserve discussion beyond what is provided in Crop Evapotranspiration.


\subsubsection{$T_{max}$ and $T_{min}$ }

Temperature data in TMY3 is recorded every hour, and used in the FAO
Penman-Monteith equation to calculate saturation vapor pressure $e_{s}$.
To calculate $T_{max}$, we record the daily high for each day in
a given month, and then compute the average of those daily highs.\[
\frac{\sum_{1}^{d}\mbox{(daily high temperature)}_{d}}{d}=T_{max}\]
 where $d$ is the number of days in the month. Substituting the daily
low temperature for the daily high temperature leads to the equation
for the average daily low.


\subsubsection{Soil Heat Flux Density ($G$) }

Some portion of solar radiation is utilized in heating the soil. $G$
is positive when the soil is warming and negative when the soil is
cooling. The soil heat flux density for a given month depends on whether
the average temperature during that month is greater or less than
the average temperature of the previous month. For example, the average
temperature in June is typically greater than the average temperature
in May, so the net soil heat flux is positive. The temperature in
November is typically cooler than the temperature in October, so the
soil heat flux is negative. We compute average temperature for each
month by state. Because we have a single calendar month, for February
through December, G is computed using average temperature as described
above. For January, we use average temperature data from the following
(prior?) December.


\subsubsection{$T_{dew}$}

Dew point temperature is used in the calculation of actual vapor pressure.
To calculate the average dew point temperature for a given month,
we take the average of all the dew point readings, which are taken
every hour:

\[
\frac{\sum_{1}^{h}(\mbox{hourly dew point temperature)}_{h}}{h}=T_{dew}\]
 where $h$ is the number of hours in the month, and $\mbox{(hourly dew point temperature)}_{h}$
is the dew point temperature at hour $h$.


\subsubsection{Elevation ($z$) }

Elevation is used in the calculation of the psychrometric constant
($\gamma$), which relies intermediately on atmospheric pressure.
Because all weather stations in a given state are averaged together,
the resulting {}``master weather station'' is given an elevation
calculated from the unweighted averages of all of that state's weather
stations.

\[
\frac{\sum_{1}^{s}\mbox{elevation}_{s}}{s}=z\]
 where $s$ is the number of weather stations in the state, and $\mbox{elevation}_{s}$
is the elevation of station $s$.


\subsubsection{Wind speed ($u_{2}$)}

The FAO Penman-Monteith Equation \eqref{eq:Penman-Monteith} relies
directly on wind speed in both the numerator and the denominator.
Hourly wind speed for a given weather station is averaged over each
day, and wind speed for a given month is the average of its constituent
days. Therefore a month's wind speed is effectively the average of
all of its hourly wind speed recordings.

\[
\frac{\sum_{1}^{h}(\mbox{hourly wind speed})_{h}}{h}=u_{2}\]
 where $h$ is the number of hours in the month and $(\mbox{hourly wind speed})_{h}$
is the wind speed during hour $h$.


\subsubsection{Latitude }

Latitude is used in the calculation of net radiation $R_{n}$. Like
elevation, latitude is averaged together over all weather stations
to come up with a latitude for the state's hypothetical master weather
station.

\[
\frac{\sum_{1}^{s}\mbox{latitude}_{s}}{s}=latitude\]
 where $s$ is the number of weather stations in the state, and $\mbox{latitude}_{s}$
is the latitude of weather station $s$.


\subsubsection{Sunshine duration ($n$) }

Net radiation ($R_{n}$) relies on a ratio of actual sunshine duration
($n$ hours) to possible sunshine duration ($N$ hours). A day with
no cloud cover would have an $n:N$ ratio of 1, with an increase in
cloud cover reducing this ratio. The World Meteorological Organization
(WMO) defines sunshine duration during a given period as the sum of
that sub-period for which the direct solar irradiance exceeds $120\frac{W}{m^{2}}$
\cite{wmo2008}. In a given day, there may be 14 or more hours (depending
on latitude) during which direct solar irradiance exceeds this number,
or as few as zero. For a particular day, $n$ is the number of hours
for which direct solar irradiance is greater than $120\frac{W}{m^{2}}$.
Because TMY3 provides direct solar irradiation data, monthly sunshine
duration ($n$) is calculated as the average of a given month's daily
sunshine durations.

\[
\sum_{1}^{d}(\mbox{sunshine duration})_{d}=n\]
 where $d$ is the number of days in the month, and $(\mbox{sunshine duration})_{d}$
is the number of hours during day $d$ during which direct solar irradiance
was greater than $120\frac{W}{m^{2}}$.


\subsection{Calculations}

The primary input variables listed above are calculated by state/subregion
($n=55$ subregions$ $) and by month ($n=12$) for a total of 660
rows. These are fed to an Excel spreadsheet which performs the intermediary
calculations outlined in Crop evapotranspiration. Those intermediary
calculations on the TMY3 variables produce the terms in the Penman-Monteith
equation, which finally calculates $ET_{0}$.

\[
\]


\bibliographystyle{plain} \nocite{*} \bibliographystyle{plain}
\bibliography{C:/Users/arsturges/Documents/LDRD/sturges_reference_database}

\end{document}
